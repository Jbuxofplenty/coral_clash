\documentclass[10pt,landscape]{article} % Landscape orientation (16in > 8in)

% === PACKAGES ===
\usepackage[
    paperwidth=3.8in, % Custom paper width
    paperheight=7.7in,  % Custom paper height
    landscape,
    margin=0.15in,       
    headheight=0pt,
    headsep=0pt,
    % footskip=0.5in % Adjust footskip if needed within margin
]{geometry}
\usepackage{multicol} % To create four columns (panels) side-by-side
\usepackage{graphicx} % For potential images
\usepackage{enumitem} % For customizing lists

\usepackage{wallpaper}

\usepackage{sectsty} % To control section font sizes
% \usepackage{lipsum} % Placeholder text - REMOVE in final version

% === STYLING ===
\pagestyle{empty} % No page numbers
% \setlength{\columnsep}{0.3in} % Adjust space between panels if needed
\setlength{\parindent}{0pt} % No paragraph indentation
% \setlength{\parskip}{4pt plus 2pt minus 2pt} % Slightly reduced space between paragraphs for tighter fit
% \sectionfont{\normalsize\bfseries} % Adjust section font size for smaller panels
% \subsectionfont{\small\bfseries} % Adjust subsection font size
% \renewcommand{\familydefault}{\sfdefault} % Use a sans-serif font (optional)
% \setlist{nosep, leftmargin=*} % Compact lists for space saving

\renewcommand{\seriesdefault}{bx}
\fontfamily{palatino}


% === DOCUMENT START ===
\begin{document}

\raggedbottom

% \vspace*{\fill}

% \includegraphics[width=\textwidth]{images/icons/title.png}

% \vspace*{\fill}

% \newpage

% \section*{Objective}
% In order to win a game of Coral Clash you must \newline complete one of the following goals:
% \begin{itemize}
%     \item \textit{Checkmate:} Threaten the opponent's Whale with capture, and it has no legal move \newline to escape. The threatened player must \newline remove the threat if possible. If a Whale is in check, it must be made safe on the next move.
%     \item \textit{Most Area Controlled:} Count up the amount of coral of your color that is not occupied by one of your opponents pieces. Triggered when any of the below three conditions is met:
%         \begin{itemize}[label=\textendash]
%             \item A player has placed all their Coral
%             \item A player only has their Whale remaining
%             \item A Crab or Octopus reaches the row of the board closest to their opponent
%         \end{itemize}
% \end{itemize}

% The game can also end in a draw or resignation.
% \begin{itemize}
%     \item \textit{Stalemate:} The player whose turn it is has no legal moves, and their Whale is not in check.
%     \item \textit{Threefold Repetition:} The exact same board \newline position occurs three times with the same player to move.
% \end{itemize}


% \section*{Setup}
% \begin{enumerate}
%     \item Place the board so each player has a green square on their right side
%     \item Collect your pieces from the box:
%         \begin{itemize}
%             \item 17 Coral
%             \item 17 Pieces: 1 Whale, 2 Dolphins, \newline 4 Turtles, 2 Pufferfish, 4 Crab, and \newline 4 Octopus
%             \begin{itemize}
%             \item Pieces that do not have four Coral icons \includegraphics[width=0.05\columnwidth]{images/icons/coral.png}  are \textit{Hunters}. 
%             \item Pieces that have four Coral icons \includegraphics[width=0.05\columnwidth]{images/icons/coral.png} \newline are \textit{Gatherers}. 
%             \end{itemize}
%         \end{itemize}

%     \item Position the pieces on the board as follows:
%     \begin{enumerate}
%         \item Pufferfish in the corners -- Hunter \newline Pufferfish on the left, Gatherer Pufferfish on the right
%         \item Hunter Turtles next to the Pufferfish
%         \item Gatherer Turtles next to the \newline Hunter Turtles
%         \item Whale occupies the two remaining squares in the middle
%         \item Dolphins occupy the two squares in the row above the Whale -- Hunter Dolphin on the left, Gatherer Dolphin on the right
%         \item Crab and Octopus are placed in the \newline following pattern in front of all the other pieces -- Hunter Crab, Gatherer \newline Octopus, Gatherer Crab, Gatherer \newline Octopus, Hunter Octopus, Hunter Crab, Hunter Octopus, Gatherer Crab 
%         \item Place a Coral under each Octopus in the third row (for a total of two)
%         \item The blue player places an additional Coral underneath the Hunter Crab that is to the right of the Gatherer Dolphin
%     \end{enumerate}
% \end{enumerate}

% \hspace{0.8cm}
% \includegraphics[width=0.85\columnwidth]{images/board.png}



% \section*{Turn Overview}
% \begin{enumerate}[label=\arabic*.]
%     \item Yellow player goes first. Players alternate turns.
%     \item On your turn, move one piece.
%     \item Resolve any Capture, Hunter Movement, \newline Hunter Effect, or Gatherer Effects that result from the move.
%     \item Check if any Objectives are met.
% \end{enumerate}

% \section*{Piece Movement}
% Pieces cannot move through other pieces.  

% \subsection*{\includegraphics[width=0.08\columnwidth]{images/icons/crab.png}  Crab} Moves one square vertically or horizontally.

 

% \begin{center}
%         \includegraphics[width=0.4\columnwidth]{images/crab.png}
% \end{center}

% % 

% \subsection*{\includegraphics[width=0.08\columnwidth]{images/icons/octopus.png} Octopus} Moves one square diagonally. 
% \begin{center}
%         \includegraphics[width=0.4\columnwidth]{images/octopus.png}
% \end{center}

% \subsection*{\includegraphics[width=0.08\columnwidth]{images/icons/puffer.png} Pufferfish} Moves any number of squares diagonally.
% \begin{center}
%         \includegraphics[width=0.4\columnwidth]{images/puffer.png}
% \end{center}


% \subsection*{\includegraphics[width=0.08\columnwidth]{images/icons/turtle.png} Turtle} Moves any number of squares vertically \newline or horizontally.
% \begin{center}
%         \includegraphics[width=0.4\columnwidth]{images/turtle.png}
% \end{center}

% \subsection*{\includegraphics[width=0.08\columnwidth]{images/icons/dolphin.png} Dolphin} Moves any number of squares vertically, \newline horizontally, or diagonally.
% \begin{center}
%         \includegraphics[width=0.4\columnwidth]{images/dolphin.png}
% \end{center}



% \subsection*{\includegraphics[width=0.12\columnwidth]{images/icons/whale.png} Whale}
% Choose one the two moves below. The Whale may never end its movement in check, where it could be captured; however, the Whale can move through a threatened square. 
% \begin{itemize}
%     \item Move half the piece any number of squares \newline vertically, horizontally, or diagonally. 
%     \begin{center}
%         \includegraphics[width=0.4\columnwidth]{images/whale.png}
%     \end{center}
    
%     \item Rotate half of the piece to one square \newline vertically or horizontally.
%     \begin{center}
%         \includegraphics[width=0.5\columnwidth]{images/whale_flip_1.png}
%     \end{center}
    
%     or
    
%     \begin{center}\includegraphics[width=0.5\columnwidth]{images/whale_flip_2.png}
%     \end{center}
    
% \end{itemize}



% Example: Whale moving through a \newline threatened square.

% \vspace{0.2cm}

% \hspace{2.55cm}
%         \includegraphics[width=0.5\columnwidth]{images/move_through_threaten.png}

% \newpage

% Example: Whale putting another Whale into check.
% \begin{center}
%         \includegraphics[width=0.5\columnwidth]{images/whale_check.png}
% \end{center}


% \section*{Capture}
% A piece captures an opponent's piece by moving to the square occupied by the enemy piece. Remove the \newline opponent's piece from the board. 

% \begin{center}
%         \includegraphics[width=0.5\columnwidth]{images/basic_capture.png}
% \end{center}

% Whales can potentially capture two pieces with the \newline same movement.

% \begin{center}
%         \includegraphics[width=0.5\columnwidth]{images/double_capture.png}
% \end{center}



% \section*{Hunter Movement}

% Pieces that do not have four Coral icons \includegraphics[width=0.05\columnwidth]{images/icons/coral.png} \newline are \textit{Hunters}. 

% When Hunters move to a square with Coral, \newline the movement ends. 

% \begin{center}
%         \includegraphics[width=0.5\columnwidth]{images/hunter_movement.png}
% \end{center}

Whales that are already on Coral with half of the piece are not stopped when the other half of the piece moves onto the same Coral.

\begin{center}
        \includegraphics[width=0.75\columnwidth]{images/whale_special_move.png}
\end{center}


\section*{Hunter Effect}
When Hunters move to a square with Coral, the piece can remove the Coral from the board, \newline returning it to the player for future use. 

\begin{center}
        \includegraphics[width=0.5\columnwidth]{images/hunter_effect.png}
\end{center}

Whales can remove two Coral in the same move

\begin{center}
        \includegraphics[width=0.5\columnwidth]{images/double_remove.png}
\end{center}



or just one Coral if desired.

\begin{center}
        \includegraphics[width=0.5\columnwidth]{images/single_remove.png}
\end{center}


\section*{Gatherer Effect}
Pieces that have four Coral icons \includegraphics[width=0.05\columnwidth]{images/icons/coral.png} are \textit{Gatherers}. 

When Gatherers move to a square without Coral, the piece can place Coral on the square.

\begin{center}
        \includegraphics[width=0.5\columnwidth]{images/gatherer_effect.png}
\end{center}




\end{document}
% === DOCUMENT END ===
